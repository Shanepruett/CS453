\begin{DoxyAuthor}{Author}
you you
\end{DoxyAuthor}

\begin{DoxyPre}\end{DoxyPre}



\begin{DoxyPre}\hyperlink{README_8dox}{README.dox}      This file
doxygen-config  Sample config file for doxygen
\hyperlink{error_8c}{error.c}         Error handling code used in class examples
loop.ca         Simple infinite loop program that doesn't use significant resources
Makefile        Build file for this example folder
TestCases       An incomplete list of test cases
test-harness/   An example (incomplete) test harness for the dash project
\hyperlink{test-readline_8c}{test-readline.c} Example file on how to use auto completion with readline library
valgrind.supp   Example suppression file for valgrind to avoid lots of spurious error messages about the readline library\end{DoxyPre}



\begin{DoxyPre}\subsection*{Readline
}\end{DoxyPre}



\begin{DoxyPre}\end{DoxyPre}



\begin{DoxyPre}See example file \hyperlink{test-readline_8c}{test-readline.c}.\end{DoxyPre}



\begin{DoxyPre}\subsection*{Valgrind
}\end{DoxyPre}



\begin{DoxyPre}\end{DoxyPre}



\begin{DoxyPre}Use valgrind as follows\end{DoxyPre}



\begin{DoxyPre}valgrind --leak-check=yes --suppressions=valgrind.supp dash\end{DoxyPre}



\begin{DoxyPre}You will need the suppression file valgrind.supp that suppresses errors from
the readline library so you can focus on issues emanating from your code.\end{DoxyPre}



\begin{DoxyPre}\subsection*{Documentation
}\end{DoxyPre}



\begin{DoxyPre}\end{DoxyPre}



\begin{DoxyPre}Generate documentation using doxygen tool. Use\end{DoxyPre}



\begin{DoxyPre}make dox\end{DoxyPre}



\begin{DoxyPre}to trigger doxygen. Use the sample doxygen-config file for using with your
project. Note that, just like javadocs, you can use any HTML tags in your
comments. All javadoc tags and comments are supported by doxygen.
\end{DoxyPre}
 